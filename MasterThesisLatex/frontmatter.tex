% !TeX encoding = UTF-8
% !TeX root = MAIN.tex

{%
	\selectlanguage{naustrian}
	\chapter*{Kurzfassung}
	
	Diese Arbeit revolutioniert die Quantifizierung von Unsicherheiten durch die Einführung von Conditional Density Methods (CDMs), einem vereinheitlichten Framework, das Conditional Density Estimation (CDE), Quantile Regression (QR) und Conformal Prediction (CP) umfasst. Wir beweisen, dass diese Techniken hinsichtlich der Einschränkung der Menge möglicher bedingter Wahrscheinlichkeitsdichtefunktionen grundlegend äquivalent sind. Aufbauend auf dieser Erkenntnis entwickeln wir neuartige Methoden zur Identifizierung optimaler konformer Regionen und zur Quantifizierung epistemischer Unsicherheiten. Umfangreiche empirische Studien auf verschiedenen Datensätzen zeigen eine State-of-the-Art-Leistung und signifikante Verbesserungen gegenüber bestehenden Ansätzen. Die vorgeschlagenen Methoden haben weitreichende Auswirkungen auf zuverlässige Entscheidungsfindungen in kritischen Bereichen wie Gesundheitswesen und Finanzen, insbesondere da unser theoretisches Framework CDE, QR und CP eine deutlich höhere Interpretierbarkeit verleiht. Diese Arbeit legt den Grundstein für zukünftige Fortschritte in der Quantifizierung von Unsicherheiten und eröffnet spannende Forschungsmöglichkeiten. Durch die Vereinigung von Theorie und Praxis stellt diese Arbeit einen bedeutenden Meilenstein in unserem Streben nach präziser und zuverlässiger Quantifizierung von Unsicherheiten dar.
}

{%
	\selectlanguage{english}
	\chapter*{Abstract}
	
	This thesis revolutionizes uncertainty quantification by introducing Conditional Density Methods (CDMs), a unified framework encompassing Conditional Density Estimation (CDE), Quantile Regression (QR), and Conformal Prediction (CP). We prove that these techniques are fundamentally equivalent in terms of restricting the set of possible conditional probability density functions. Building upon this insight, we develop novel methods for identifying optimal conformal regions and quantifying epistemic uncertainty. Extensive empirical studies on diverse datasets demonstrate state-of-the-art performance and significant improvements over existing approaches. The proposed methods have wide-ranging implications for reliable decision-making in critical domains such as healthcare and finance, in particular since our theoretical framework gives significantly more interpretability to CDE, QR and CP. This work lays the foundation for future advancements in uncertainty quantification and opens up exciting research avenues. By unifying theory and practice, this thesis represents a major milestone in our pursuit of accurate and reliable uncertainty quantification.
}